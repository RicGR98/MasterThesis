\chapter{Experiments}


\section{Introduction}

As we have seen in the previous section, all results of a simulation are stored into several \texttt{.csv} files.

\begin{itemize}
    \item \texttt{states.csv} contains, for each State, its id, population size, VAT rate, levy rate, tariff rate, wealth tax rate, allowance type, unemployment rate, black economy share, GDP, the money it has, the total money of its population (agents), the number of transactions performed by its agents, and the id's of the other States to which it is connected.

    \item \texttt{agents.csv} contains, for each Agent, its id, the id of its State, its initial money (the money with each it started the simulation), its current money, its talent, whether it is a producer or not (thus whether it is employed or unemployed) and the number of purchases it has made.
    
    \item \texttt{products.csv} contains, for each Agent, the id of the Agent producing it, its type, the production price (always set to 0), the selling price, the stock of this product, and the number of sales. Therefore, each agent has one Product and we can merge this file with the agents file.
    
    \item \texttt{ticks.csv} keeps track of different values during the simulation. The following metrics are saved every 100 ticks: the number of ticks that have passed so far, the number of transactions that have happened in the World so far, the total money of all the States, the total money of all the Agents, and the total GDP of all States.
\end{itemize}

Based on these files, we can do various experiments by adjusting the parameters and see the effects on some key metrics. For this, the language Python3 has been chosen as it contains many libraries to analyze and plot such data: pandas, numpy or matplotlib. Rapidness is not the key factor here as the simulation has already stored its results in the \texttt{.csv} files.

The default configuration file that we use is the following. Experiment after experiment, we will modify one or several parameter and see its/their influence. 

\begin{lstlisting}[language=json,firstnumber=1]
{
    "World": {
        "PRODUCT_CHOICE": "CHEAPEST",
        "NB_STATES": 200,
        "NB_AGENTS": 30000,
        "NB_TICKS": 4000,
        "NB_TICKS_SAVE_CSV": 100
    },
    "Connections": {
        "CLUSTER_SIZE": 0,
        "PROB_CONNECTION": 0.0
    },
    "State": {
        "Tax": {
            "MIN_VAT": 0.2,
            "MAX_VAT": 0.2,
            "MIN_LEVY": 0.1,
            "MAX_LEVY": 0.1,
            "MIN_TARIFF": 0.3,
            "MAX_TARIFF": 0.3,
            "VAL_WEALTH_TAX_TOP": 0.1,
            "MIN_WEALTH_TAX_VALUE": 0.2,
            "MAX_WEALTH_TAX_VALUE": 0.2,
            "NB_TICKS_COLLECT_TAXES": 100
        },
        "Allowance": {
            "NB_TICKS_DISTRIBUTE_ALLOWANCES": 100
        },
        "Others": {
            "MIN_UNEMPLOYMENT": 0.05,
            "MAX_UNEMPLOYMENT": 0.05,
            "MIN_BLACK": 0.15,
            "MAX_BLACK": 0.15
        }
    },
    "Agent": {
        "MIN_INIT_MONEY": 1000.0,
        "MAX_INIT_MONEY": 1000.0,
        "RATIO_BUY": 0.5,
        "RATIO_PRODUCE": 0.5
    },
    "Product": {
        "NB_DIFF_PRODUCTS": 50,
        "MIN_PRICE": 0.0,
        "MAX_PRICE": 300.0,
        "MAX_STOCK": 2000
    }
}
\end{lstlisting}

Whenever points are scattered on a plot, we will also plot a line which interpolates, with a degree 2, these points to better highlight the trend. Also, in the bar charts, an average is computed in order to give a global overview. We also have to pay attention that most plots have two $y$ axis (red is for the left hand-side, blue for the right hand-side). There are over 40 plots in the folder \texttt{./report/img/exp/}, we will only focus on the most interesting ones.

For each experiment, we will compare it with the state-of-the-art presented earlier, and try to answers the many research questions introduced in the section~\ref{section:motivation_objectives}.


\section{State experiments}

We will start our experiments with all the parameters related to the State. We should note that to be able to compare fairly different States, the metrics of the State (for instance the total money of its Agents or the GDP) is divided by its population size. 

    \subsection{Experiment 1: No taxes}
    In this first experiment, we will try to see the effects on having no taxes (thus all tax values are set to 0). In the following plots, we will plot the State with regular taxes (as defined earlier) on the left hand-side, and the State with no taxes on the right hand-size. 

        \subsubsection{State GDP and number of transactions}

        We can see on the following plots that a State with no taxes (thus VAT of 0, but the other taxes are also set to 0) will generally have a smaller GDP and the number of transactions is also decreased compared to a State with taxes (left plot where the VAT is 0.2 for instance). 
        
        At first this might seem rather odd since we expect products to be cheaper, therefore more transactions should be happening (hence the GDP would be boosted too). However, if we think more about it, this seems rather logical because a State with no tax will not be able to distribute allowances, and thus some agents will concentrate all the money at the expense of other agents who will not be able to buy products anymore after some time, eventually diminishing both the GDP and the number of transactions.

        \begin{figure}[H]
            \minipage{0.5\textwidth}
                \includegraphics[width=\linewidth]{img/exp/1_1_1.png}
                \caption{Normal taxes (e.g. VAT of 0.2)}
            \endminipage\hfill
            \minipage{0.5\textwidth}
                \includegraphics[width=\linewidth]{img/exp/1_2_1.png}
                \caption{No taxes (e.g. VAT of 0)}
            \endminipage\hfill
        \end{figure}

        \subsubsection{Gini coefficient} 
        
        Another interesting metric is the measure of inequalities, i.e. the Gini coefficient, the smaller it is, the more equal a society is. We can, naturally, see a tremendous difference between the two plots. Indeed, if the State has no taxes, it cannot collect nor distribute money. Therefore, the Gini coefficient skyrockets from $0.39$ to $0.96$.

        \begin{figure}[H]
            \minipage{0.5\textwidth}
                \includegraphics[width=\linewidth]{img/exp/1_1_3.png}
                \caption{Normal taxes (e.g. VAT of 0.2)}
            \endminipage\hfill
            \minipage{0.5\textwidth}
                \includegraphics[width=\linewidth]{img/exp/1_2_3.png}
                \caption{No taxes (e.g. VAT of 0)}
            \endminipage\hfill
        \end{figure}

        This experiment shows the importance of having taxes (in the broadest sense of the term) on several metrics in order for our society to advance and be more fair.

    \subsection{Experiment 2: VAT}
    We will now, for each of the four taxes that were presented, analyze their influence. First: the VAT. For this, we will generate many States with random values for the VAT (ranging from 0 to 1) and see if we can see any pattern emerging regarding some metrics.

        \subsubsection{State GDP and number of transactions}

        \begin{wrapfigure}{l}{0.5\linewidth}
            \includegraphics[width=\linewidth]{img/exp/2_1.png}
        \end{wrapfigure} 
        { \lipsum[1-2] %TODO
        \par

        \subsubsection{Gini coefficient}

        \begin{wrapfigure}{r}{0.5\linewidth}
            \includegraphics[width=\linewidth]{img/exp/2_3.png}
        \end{wrapfigure} 
        { \lipsum[1-2] %TODONaturally, as we had seen before, the lower the VAT is, the more inequalities we will have since the State has less money to redistribute fairly. However, we can see that here, with a VAT of 0, the Gini coefficient is around 0.45 whereas it was at 0.91 before. This is because in this case, the other taxes are still present as shown in the default configuration allowing the State to distribute allowances and diminishing inequalities. Hence why the higher the VAT is, the less inequalities we have since the Gini coefficient tends to go to zero.
        \par

    \subsection{Experiment 3: Tariff}
    As for the VAT, we will now analyze the influence of the tariff tax by generating many States with random values (ranging from 0 to 1) and see if we can see any pattern emerging regarding some metrics. 

        \subsubsection{State GDP and number of transactions}

        \begin{wrapfigure}{l}{0.5\linewidth}
            \includegraphics[width=\linewidth]{img/exp/3_1.png}
        \end{wrapfigure} 
        { \lipsum[1-2] %TODO
        \par

        \subsubsection{Gini coefficient}

        \begin{wrapfigure}{r}{0.5\linewidth}
            \includegraphics[width=\linewidth]{img/exp/3_3.png}
        \end{wrapfigure} 
        { \lipsum[1-2] %TODO
        \par


    \subsection{Experiment 4: Levy}
    We will now analyze the influence of the levy tax by generating many States with random values (ranging from 0 to 1) and see if we can see any pattern emerging regarding some metrics. 

        \subsubsection{State GDP and number of transactions}

        \begin{wrapfigure}{l}{0.5\linewidth}
            \includegraphics[width=\linewidth]{img/exp/4_1.png}
        \end{wrapfigure} 
        { \lipsum[1-2] %TODO
        \par

        \subsubsection{Gini coefficient}

        \begin{wrapfigure}{r}{0.5\linewidth}
            \includegraphics[width=\linewidth]{img/exp/4_3.png}
        \end{wrapfigure} 
        { \lipsum[1-2] %TODO
        \par


    \subsection{Experiment 5: Wealth tax}
    We will now analyze the influence of the last tax: the wealth tax with the same methodologies as the previous taxes and see if we can see any pattern emerging regarding some metrics. 

        \subsubsection{State GDP and number of transactions}

        \begin{wrapfigure}{l}{0.5\linewidth}
            \includegraphics[width=\linewidth]{img/exp/5_1.png}
        \end{wrapfigure} 
        { \lipsum[1-2] %TODO
        \par

        \subsubsection{Gini coefficient}

        \begin{wrapfigure}{r}{0.5\linewidth}
            \includegraphics[width=\linewidth]{img/exp/5_3.png}
        \end{wrapfigure} 
        { \lipsum[1-2] %TODO
        \par


    \subsection{Experiment 6: Unemployment}
    After analyzing the different taxes, we will now focus on another parameter which cannot directly be controlled by the State. As usual, we will create many States with different unemployment rates and see how our metrics change.

            
        \subsubsection{State GDP and number of transactions}

        \begin{wrapfigure}{l}{0.5\linewidth}
            \includegraphics[width=\linewidth]{img/exp/6_1.png}
        \end{wrapfigure} 
        {As one could have expected, we have a rather clear linear correlation. Indeed, the higher the unemployment rate is, the less number of transactions we have and the lower the GDP is. Actually, it is a vicious cycle because the less producers we have, the less products are available on the market, thus less transactions. By having less transactions, we collect less taxes, and therefore do not distribute as much, this means that some agents will never be able to buy another product. 
        \par

        \subsubsection{Gini coefficient}

        \begin{wrapfigure}{r}{0.5\linewidth}
            \includegraphics[width=\linewidth]{img/exp/6_3.png}
        \end{wrapfigure} 
        {The plot on the side is a bit tricky but more interesting to analyze. If we only look at the scattered points, we see that there is a linear correlation from the rate 0.0 until 0.8: the more non-producing agents we have, the bigger the Gini coefficient is. This is logical because producing agents will still be able to sell their products and make some money thus accumulating more wealth compared to others, increasing the inequalities.

        However, after the 0.8 rate on the $x$ axis, we can see a significant drop, and the previous statement does not hold true anymore: the Gini coefficient gets very close to zero, i.e. perfect equality. Actually, it also makes sense because if almost everybody is unemployed, then no one can afford to buy any product, thus we have almost no transactions happening and no money flowing between agents, and no agent can therefore become richer than others.
        \par


    \subsection{Experiment 7: Black economy}
    The second parameter which cannot directly be controlled by the State is the black economy. As usual, we will create many States with different share of black economy happening and see how our metrics change.

            
        \subsubsection{State GDP and number of transactions}

        \begin{wrapfigure}{l}{0.5\linewidth}
            \includegraphics[width=\linewidth]{img/exp/7_1.png}
        \end{wrapfigure} 
        {Until now, we were used to see a strong relation between the number of transactions and the State's GDP. However, this is not the case anymore because of the black economy (because even when no due tax is paid, the transaction is still counted). This results in a linear correlation between the black economy share and the GDP as expected, yet we see absolutely no correlation with the number of transactions since the blue line is quite straight.
        \par

        \subsubsection{Gini coefficient}

        \begin{wrapfigure}{r}{0.5\linewidth}
            \includegraphics[width=\linewidth]{img/exp/7_3.png}
        \end{wrapfigure} 
        {As the black economy share increases, the inequalities augment as well. This is simply due to the fact that the State will have less money to redistribute when giving allowances, therefore inequalities will subsist even after the small amount of money collected as been redistributed. \\ \\
        \par
    


    \subsection{Experiment 8: Allowances}
    As we have seen, we have two types of allowances: the flat one (which mimics the Universal Basic Income in the sense that the money is distributed regardless of the agent's wealth), and the fair one (based on the agent's wealth). We will analyze the effects of these two types of redistribution mechanisms.
    
        \subsubsection{State GDP and number of transactions}

        \begin{wrapfigure}{l}{0.5\linewidth}
            \includegraphics[width=\linewidth]{img/exp/8_1.png}
        \end{wrapfigure} 
        { On this side plot, we see that the flat redistribution mechanism seems to be \emph{slightly} superior to the fair one. The increase is of about $2\%$ (2.1\% for the GDP going from $19988$ to $20421$ and $1.7\%$ for the number of transactions going from 679 to 691). This increase is rather negligible and might be the result of the stochastic behavior of the simulation. Indeed, one would expect to not be a relevant difference on these metrics. For instance, the number of transactions is not penalized by the Flat allowance because poorer agents will still be able to afford buying products since they receive their (equal) share of the total money distributed: all Agents are still able to buy products,.
        \par

        \subsubsection{Gini coefficient}

        \begin{wrapfigure}{r}{0.5\linewidth}
            \includegraphics[width=\linewidth]{img/exp/8_3.png}
        \end{wrapfigure} 
        { On the contrary, one would expect a quite pertinent difference on the metric of inequalities as depicted on the plot. This results seems logical since it is was the \emph{main goal} of this redistribution system: State distributing the collected money in a fair way have, in average, a smaller Gini index (i.e. inequalities are limited) and vice-versa for the flat distribution mechanism. With this flat allowance, the wealthier agents will stay wealthier.

        However, an important note to make is that both these systems clearly outperform the system where no redistribution is in place. Indeed, we had seen in the first experiment that, in this case, the Gini coefficient would skyrocket to $0.96$.
        \par



    \subsection{Experiment 9: Connections}
    The final parameter of States that will be analyzed is the number of Connections it has. For this we will study both the bilateral connections and the cluster and their influence on our already well presented metrics. In this experiment, we have a probability of connection of 0.3 and a cluster of 5 States (thus each State is connected to 4 others).

        \subsubsection{State GDP and number of transactions}

        \begin{wrapfigure}{l}{0.5\linewidth}
            \includegraphics[width=\linewidth]{img/exp/9_1.png}
        \end{wrapfigure} 
        { First, as the number of connected states increases, it seems that the average GDP of these states decreases. Although this may seem odd at first, it fits with our implementation because agents will tend to buy cheaper products since they have a wider range of choices. On the other side, the number of transactions remains quite constant (besides the small drop when a State is connected to 3 others which might be due to the fact that not many States have 3 connections and therefore the statistical relevance is biased).

        With these metrics going towards opposite directions, we can notice something very interesting: the difference between the red (GDP) and blue (number of transactions) bars keeps increasing. Indeed, at first, when a State is connected to no other State, these two metrics are very related as we have seen up until now (besides in the black economy experiment). However, the difference is much more noticeable when a State is connected to 4 other States. The main hypothesis for this behavior is that agents perform as many transactions as before, however because they have access to other markets without any tariffs, they are able to buy some products at a cheaper price in another State therefore not increasing the GDP as much.

        The results of this experiment are quite lackluster when compared to the state of the art. Indeed, an increase of the GDP was expected for State with more connections. This difference is due to the code implementation which does not mimic well enough the reality, and not to the simulation itself. The simulation results fit the implementation.
        \par

        \subsubsection{Gini coefficient}

        \begin{wrapfigure}{r}{0.5\linewidth}
            \includegraphics[width=\linewidth]{img/exp/9_3.png} 
        \end{wrapfigure}   
        { Regarding the measure of inequalities, we see that all the Gini coefficients are very similar to one another. The differences are quite negligible when compared to the others when had before (0.46 versus 0.55, or 0.39 versus 0.96). Having access to a larger market with cheaper prices does not, in fine, mean that inequalities will be reduced/accentuated for those agents since both the wealthiest and the poorest agents will buy these cheaper products, therefore no reduction of difference of wealth was expected. 
        \par
    

    \subsection{Experiment 10: Number of ticks}
        
        \begin{wrapfigure}{l}{0.5\linewidth}
            \includegraphics[width=\linewidth]{img/exp/10_2.png}
        \end{wrapfigure} 
        { Our last experiment of this section is not based on a parameter of the state \emph{per se}, but on the duration of the simulation and the evolution of some State metrics as the number of ticks increases. Deciding how long the simulation should run for is important to save computational power and run-time by checking at around how many ticks the simulation starts to stabilize. After this point of stabilization, there is no real advantage of running the simulation for even longer because no dramatic fluctuation should happen. 
            
        As we can see on this plot, the simulation metrics start to stabilize after 2000 or 3000 ticks and going up to 6000 ticks provides no advantage (for a run-time 2-3 times longer). Hence, the default number of ticks in the configuration file is 3000.
        \par
    
\section{Agent experiments}
    After seeing and analyzing some State parameters, we will focus on those of the Agents.

    \subsection{Experiment 11: Talent}
    The talent an agent has lets it produce cheaper products therefore having more chances at getting picked by a potential buyer looking for the cheapest product. For this experiment, we will analyze how the talent of an agent (between 0 and 1) influences its sales and purchases by generating lots of homogeneous agents with different talent values.

        \begin{wrapfigure}{r}{0.5\linewidth}
            \includegraphics[width=\linewidth]{img/exp/11.png}
        \end{wrapfigure} 
        { This first plot may look quite chaotic and random at first, however we can see many interesting patterns by taking into account some particularities afterwards. First, generally we see a trend (red line behind the blue dots) that the more talent an agent has, the more sales it will make. Regarding the number of purchases, we have a rather odd pattern (which looks like the on in the Gini coefficient in the Unemployment experiment): as the talent increases, we are gradually able to make more purchases until a certain point around 0.5-0.7. After this point, there is a significant drop due to the fact that agents will produce products ``too cheap'', therefore not allowing them to earn enough wealth and make purchases. This is quite interesting because it shows that selling at such low prices is not always a good idea.
        
        However, it seems that the trends are not linear (as one could have expected) but this due to the fact a different products have different base prices (i.e. the original price of the product which is then lowered by a certain factor based on the agent's talent). For instance, if we have an agent A which produces a product with a base price of 10 with a talent of 0.1, then A will see its product at 9. Now if we have another agent B which produces a product with a base price of 200 with a talent of 0.7, its product will now cost 60. This means that, even though A has less talent (0.1) than B (0.7), it will \emph{probably} sell more products since its product remains cheaper, therefore we do not have a clear linear trend.
        This is also the reason why we a trace of red scattered points going upwards: the same product: the point goes up (i.e. more sales) as the talent goes up too.

        \par



    
%\section{Product experiments}
\chapter{Conclusion}

Throughout this thesis, we have been exploring the role of many key parameters that are part of most economical systems around the world. For this, we have simulated an economical system (the World) by generating many homogeneous and independent agents belonging to different States with different parameters. All these agents interact with each other according to certain rules edicted by their States. 

We have seen the State-of-the-art research regarding both economics and computer science in order to be better equipped for the simulation, the experiments and the optimization. This allowed us to have a base for future comparisons, and see whether or not our results fitted the actual reality.

By letting these agents interact with one another throughout ticks (representing time passing by), we were able to see emerging patterns on certain metrics such as the Gini coefficient or the number of transactions. By modifying some initial parameters, we could influence the final metrics, therefore allowing us to understand the role of each parameter as we have seen in the \nameref{chap:Experiments} chapter. We have also compared this experiments with State-of-the-art research happening in both the computer science and economics fields. Most experiments fitted the reality depicted by these research, however, a few did not, due to the complexity of the task.

To go even further, we tried to optimize some key parameters of the State in order to minimize or maximize some metrics. This was done through the use of Particle Swarm Optimization, a population-based method, which let us optimize continuous variables method. This was done to confirm the role of each parameter and how States could improve these metrics. We have seen that these results fitted the ones we have in our experiments as well as the ones in the State-of-the-art.

As we have previously stated, this computer simulation is far from being perfect or close to reality. We tried to focus ourselves on the big picture and the essential parts of an economical system at the cost of precision. There is still a lot of room to improvement. Indeed, in future works, we could improve this by taking into account many new parameters such as the quality of a product or its carbon footprint,  create luxurious products, have multiple clusters, different types of agreements between States, use multi-objective optimization to optimize multiple metrics at the same time, ...